% !TEX root=report.tex
\section{Conclusion} \label{sec:conclusion}

To the best our knowledge this paper represents the first effort in using recommender systems and machine learning for social travel on platforms like CouchSurfing.
Our proposed system recommends surfers to hosts and vice versa in order to maximize the acceptance rate of couchrequests. This benefits both surfer and hosts: the surfer has to write less requests in order to find a place to
stay, and the host has to review less requests to find a surfer he would like to host.
We model the acceptance probability using a rich feature set including personal information like age, gender, languages spoken, countries lived in, countries traveled to, etc., their interests, and their status and activity on Couchsurfing.org which includes e.g. the number of references, the number of friends, or the response rate in the internal messaging system.
We learn what hosts prefer in couch surfers and which combinations of attributes increase their chance of getting accepted. Furthermore, we allow for host-specific personalization using the hashing trick since different hosts have 
different preferences.
We have shown that our proposed model can more accurately predict which surfer is most likely to get accepted by a given host than a random baseline and baselines similar to the currently deployed ranking system at CouchSurfing.

Future work includes an evaluation from the surfer's side, i.e. whether we can accuratly predict by which host he is going to be accepted. Further, we would like to investigate the benefits of modeling more dependencies across user properties and preferences (possibly using kernels), i.e. whether age influences preferences in gender or whether a host that wants to travel to a certain country likes to host surfers from that country. We plan to evaluate whether our recommender system leads to a higher acceptance rate in the deployed system at CouchSurfing. Using this system we also want to explore more the temporal structure in the data better, i.e. whether both parties communicated before the request, were already friends, or whether the couch was already booked for the requested time slot or still available etc. The latter information could be used to filter the training data because the only reason for the rejection might have been the unavailalability of the host or his couch.
\todo{more ideas for future work?}
